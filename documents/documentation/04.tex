\chapter{Conclusion}
\section{Review}
Our opinion about the course ''Softwareproject: Model driven software development'' divides into 2 pieces. The first part tends to an good introduction regarding the content of this seminar. We got basic knowledge about the main content in an enjoyable way and could deploy it a few weeks later. \\
The second part tends to some adverse criticism. The effort of our exercise can be divided into different parts. We really dislike the ratio of the amount of the model driven software development compared to the amount of other tasks. We spend a lot of time to get different things working and developed workarounds because of bugs.  But we required just a few hours to deploy our knowledge of model driven software development to create a smm parser. Therefore, the difficulty is not to use model driven methods, rather than to find an efficient solution of the given task. \\
In addition the whole effort of our task is unpredictable because of Sonar has nearly no documentation. Every technical issue just can be solved with the help of the devlopers mailing list. We think student projects should be a little bit of appreciable.

\section{Conclusion}
Our project was very instructive as well as ordinary. We got an insight of the Frauenhofer Institut what is additionally an gain of knowledge. Within the frame of this course we developed an Sonar plugin that works with Metrino and Modelbus. \\
Because of the leak of time we concentrated our work to developed some basic metrics, just a basic way of visualisation of defined metrics and extended sonar to understand uml. It's possible to enhance our application with additional metrics, more display modes and moreover define more languages like ecore in order that sonar can analyse it. \\ 
It's not a perfect solution but an good example how the given task could be solved and what problems will appear. The classloader problem and other limited functions of the Sonar plugin system are the centre of difficulties. In addition another disadvantage depends the huge build time of an sonar plugin above all if debug mode is activated. For time reasons we couldn't solve one requirement that depends the analyse of real code in the repository. It's not an feature of the current version of sonar and we could not found a workaround but it should be possible in the prospective versions. Furthermore we spend a lot of time in identifying interfaces and comprehend the interaction. This documented knowledge can be used for future works.
\section{Closing Words}
We want to express our appreciation to our course instructor for their effort. We have learned a lot of model driven software development and got an interesting task that intent us the whole time in a positive way. 