\chapter{Introduction}
\section{Motivation}
Software development processes are usually divided into different phases. Examples are the requirement analysis, the design phase or the testing phase. Different process models like the waterfall model, or the V-model describe this process. A special form of software developement is model driven software developement. In all phases models are in main focus. It’s an important way to deal with complexity of huge systems. 
In order to achieve a seamless integration between tools and the work of the developement team it’s important to decide, how information can be stored and reused in a transparent way. One possible approach is ModelBus. 
We come together as a team to expand ModelBus.
Our team consists of five computer science MSc students who are motivated to find a stable solution for the given task: Alexander Dümont, Arsenij Solovjev, Damla Durmaz, Ferhat Beyaz, Markus Rudolph, Sebastian Barthel.
Our goal is to expand SONAR to cooperate with ModelBus. 
Sonar is an open source software quality platform. Sonar uses various static code analysis tools such as Checkstyle, PMD, FindBugs to extract software metrics, which then can be used to improve software quality. 
It should be possible to cooperate with the ModelBus repository. Another goal is to be able to use ModelBus’s Metrino to calculate metrics on models, and eventually provide new metrics.

\section{Requirements}
The product owner identified the following requirements:
\begin{enumerate}
\item Create a plugin for sonar (closed)
\item Comprehend and cooperate with ModelBus repository (closed)
\item Transfer as many as possible metrics from Metrino to Sonar (open)
\item Select the model (closed)
\item Traverse the directory tree (closed)
\item Analyse sourcecodes (cancelled)
\item Plugin must work as efficient as possible (closed)
\end{enumerate}

\subsection{Create a plugin for sonar}
\begin{itemize}
\item Requirement Number: 1
\item Description: Each sonar plugin has a specific architecture. The created sonar plugin must hold the specifications for a sonar plugin. Additionally, it must be run on a server and offers a WSDL interface.
\item Type: functional
\item Dependencies: -:- (should be done first)
\item Priority: high
\item State: closed
\item Fit Criterion: The plugin can be build and runs on a chosen server. A client connects to the Modelbus server and understands the WSDL interface. The client receives a message from the WSDL interface of the Modelbus server.
\end{itemize}

\subsection{Comprehend and cooperate with ModelBus repository}
\begin{itemize}
\item Requirement Number: 2
\item Description: ModelBus has an own repository, where it stores artefacts of the tools, which are connected to the ModelBus via adapters. It must be understood, how ModelBus connects to the repository and how an external tool (like a sonar plugin) can do this.
\item Type: functional
\item Dependencies: \# 1
\item Priority: high
\item State: closed
\item Fit Criterion: The sonar plugin is able to connect to the ModelBus repository via a command. It can fetch the sourcecode for a random project to test, if the connection is established. The sonar plugin can terminate the connection to the respository for a given command.
\end{itemize}

\subsection{Transfer as many as possible metrics from Metrino to Sonar}
\begin{itemize}
\item Requirement Number: 3
\item Description: The sonar plugin can already fetch model code from the repository. The models are read by Metrino and analyzed by Metrino with OO metrics. The results of the analysis are fetched from metrino by the sonar plugin and visualized on the web interface. Metrino offers a lot of metrics and it should be possible to offer as many metrics from metrino as possible.
\item Type: functional, partly non-functional
\item Dependencies: \# 4
\item Priority: high
\item State: open
\item Fit Criterion: The sonar plugin sends a model to Metrino with the information which OO metric shall be applied. Metrino returns the results. The results are vizualized. The plugin offers to select at least 10 OO metrics of Metrino to apply to the sent model code.
\end{itemize}

\subsection{Select the model}
\begin{itemize}
\item Requirement Number: 4
\item Description: The sonar plugin can fetch model code from the repository and transform the model into a format which can be read by model metric analyzing tools, like Metrino.
\item Type: functional
\item Dependencies: \# 3
\item Priority: high
\item State: closed
\item Fit Criterion: The plugin fetched a model. It transforms the model into the format of Metrino. For a test case, the transformed model shall be saved in a file and read successfully by Metrino.
\end{itemize}

\subsection{Traverse the directory tree}
\begin{itemize}
\item Requirement Number: 5
\item Description: In the directory tree of sonar plugin, a lot of sourcecode and model code projects are placed. The plugin should be able to analyze all codes from the repository.
\item Type: functional
\item Dependencies: -:-
\item Priority: high
\item State: closed
\item Fit Criterion: The plugin fetched the whole project tree of the repository. It loads each project into its own directory tree. It traverses the whole directory tree. For each project in its directory tree, it analyzes the code and visualizes it on the web interface.
\end{itemize}

\subsection{Analyse sourcecodes}
\begin{itemize}
\item Requirement Number: 6
\item Description: The sonar plugin can apply a set of metrics to the fetched sourcecode. The results are visualized with graphs or statistics and can be get via the web interface.
\item Type: functional
\item Dependencies: \# 2
\item Priority: high
\item cancelled, because requirement is infeasible with current Sonar version (could be with future versions)
\item Fit Criterion: The plugin fetches normale program source code from the repository. It applies a couple of metrics, which can be chosen from the application interface.
\end{itemize}

\subsection{Plugin must work as efficient as possible}
\begin{itemize}
\item Requirement Number: 7
\item Description: Plugin must traverse the directory tree fastly. The connection to the ModelBus must provide security standards, but it should not be slow. The results of the analysis should be offerred via the web interface as fast as possible.
\item Type: non-functional
\item Dependencies: \# 6
\item Priority: middle
\item closed
\item Fit Criterion: Because of the difficulty of the decision, if it works properly fast, a usability test with the customer must be done, who can tell in a better way, if the product works fast or not.
\end{itemize}

\section{Expectations}


\section{Model-Driven Engineering}


\section{Responsibilities}


\section{Paper Purposes}
